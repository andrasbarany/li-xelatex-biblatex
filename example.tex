\documentclass{li-xelatex-biblatex}

\usepackage{lipsum}

\newcommand{\AuthorA}{%
    András Bárány\\%
    Leiden University Centre for Linguistics\\%
    Leiden University}

% You can add additional authors the following way:
% \newcommand{\AuthorB}{2nd author\\Department\\Institution}
% You can add contact information for additional authors (see end of file):
% \contact{\AuthorX}{email}

% \author{\AuthorA \and \AuthorX} for multiple authors
\author{\AuthorA}

\title{LI class example}

\makeatletter
\AtBeginDocument{\toggletrue{blx@useprefix}}
\AtBeginBibliography{\togglefalse{blx@useprefix}}
\makeatother

\begin{document}

\maketitle

\begin{abstract}\lipsum[3]\end{abstract}
\keywords{Example keyword}

\section{Section heading}\addtoendnotes{Acknowledgement endnote.}

\lipsum[1]

\section{Section heading}

\subsection{Subsection heading}

\lipsum[2]

In-text citation with page number \textcite[443--445]{EKiss2008}. \emph{LI}
requires not using parentheses when referring to a publication, for example in
phrases like \enquote{in the sense of \cite{EKiss2008}}. Finally, the following
are citations in brackets \parencite{EKiss2008,Szabolcsi1994}.

This is an in-text citation of a paper that has not appeared yet:
\textcite{Barany2020}. In parentheses, the pubstate has to be separated by a
comma \parencite{Barany2020}. \Textcite{denDikken2018}, \textcite{vanUrk2015}
shows that Dutch names with \emph{tussenvoegsels} can have different
capitalisations.

\printbibliography

\contact{\AuthorA}{a.barany@hum.leidenuniv.nl}

\end{document}
